\documentclass[11pt,a4paper]{moderncv}

% moderncv themes
\moderncvtheme[green]{classic}                  % optional argument are 'blue' (default), 'orange', 'green', 'red', 'purple', 'grey' and 'roman' (for roman fonts, instead of sans serif fonts)
%\moderncvtheme[green]{classic}                % idem

% character encoding
\usepackage[utf8]{inputenc}                   % replace by the encoding you are using
\usepackage{CJKutf8}

% adjust the page margins
\usepackage[scale=0.75]{geometry}
\recomputelengths
%\setlength{\hintscolumnwidth}{3cm}						% if you want to change the width of the column with the dates
%\AtBeginDocument{\setlength{\maketitlenamewidth}{6cm}}  % only for the classic theme, if you want to change the width of your name placeholder (to leave more space for your address details
%\AtBeginDocument{\recomputelengths}                     % required when changes are made to page layout lengths

% personal data
\firstname{李}
\familyname{晓东}



%\title{Resumé title (optional)}               % optional, remove the line if not wanted
\address{大连理工大学软件学院}{大连市经济技术开发区, 116620}    % optional, remove the line if not wanted
%\mobile{+30 698 4385057}                    % optional, remove the line if not wanted
\phone{+86 13426475724}                      % optional, remove the line if not wanted
%\fax{fax (optional)}                          % optional, remove the line if not wanted
\email{lxd.dlut@gmail.com}                      % optional, remove the line if not wanted
%\email{\href{mailto:s.dakourou@gmail.com}{s.dakourou@gmail.com}}                      % optional, remove the line if not wanted
\homepage{stephenlee.github.com}                 % optional, remove the line if not wanted
%\extrainfo{additional information (optional)} % optional, remove the line if not wanted
%\photo[64pt][0.4pt]{picture}                         % '64pt' is the height the picture must be resized to, 0.4pt is the thickness of the frame around it (put it to 0pt for no frame) and 'picture' is the name of the picture file; optional, remove the line if not wanted
%\quote{Some quote (optional)}                 % optional, remove the line if not wanted


% bibliography with mutiple entries
%\usepackage{multibib}
%\newcites{book,misc}{{Books},{Others}}

%\nopagenumbers{}                             % uncomment to suppress automatic page numbering for CVs longer than one page
%----------------------------------------------------------------------------------
%            content
%----------------------------------------------------------------------------------
\begin{document}
\begin{CJK}{UTF8}{gkai}
\maketitle

\section{教育背景}
%\cventry{year--year}{Degree}{Institution}{City}{\textit{Grade}}{Description}
\cventry{2011-- 至今}{硕士}{大连理工大学软件学院}{将于 2014 毕业} 
{研究方向: 异构网络挖掘}{感兴趣方向:机器学习,社交网络,数据分析}

\cventry{2007--2011}{本科}{大连理工大学软件学院}{} {}{} %arguments 3 to 6 are optional

\section{个人经历}

\subsection{获奖情况}
\cventry{2011--}{研究生入学一等奖}{}{}{}{}
\cventry{2009--2010}{学习类二等奖学金}{top 15\%}{}{}{}
\cventry{2007--2008}{学习类二等奖学金}{top 15\%}{}{}{}

\subsection{实习}
\cventry{2013.6.18--now}
{数据分析}{人人游戏}{中国北京}
{移动广告平台的反作弊分析,微博数据抓取,微博主题分析、好友可视化、影响力分析, 新闻内容抓取, 新闻内容summary}{Hadoop, Python,ElasticSearch}{}

\cventry{2010.6--2010.9}
{Java工程师}{阿里巴巴有限公司}{中国杭州}
{设计并实现了一个监控通知系统,包括页面抓取,关键字匹配,邮件/旺旺提醒,账户管理等模块}{Java实现}{}

\section{项目经历}
\renewcommand{\baselinestretch}{1.2}
\vspace*{0.2\baselineskip}
\cventry{2011--2012}{topic\_model}{分析新浪微博微博主的博文主题}{Python,gensim}{课程项目}{代码:\url{https://github.com/stephenLee/topic_model}}
\vspace*{0.2\baselineskip}
\cventry{2012--至今}{sobot}{基于新浪微博和人人网的机器人, 实现了定时从Reddit上抓取热门的机器学习话题然后发送到微博和人人上.微博主页:\url{http://weibo.com/reddit4ml}}{Python,Google App Engine}{个人项目}{代码: \url{https://github.com/datahacking/sobot}}
\vspace*{0.2\baselineskip}
\cventry{2012--至今}{visu}{可视化个人的人人网好友}{python,d3.js}{个人项目}{代码:\url{https://github.com/stephenLee/RenrenRecipies}}

\section{学术研究}
\cventry{2013}
{Link Prediction in Heterogeneous Networks
  Based on Tensor Factorization}{基于张量对异构网络建模,通过对真实的异构网络数据实验说明了使用CP张量分解能够更好的挖掘异构网络的结构,提高了链路预测的准确性}{Journal of Engineering Science and Technology Review}{\emph{EI}}{}

\section{语言技能}
\cvlanguage{中文简体}{母语}{}
\cvlanguage{英语}{CET-6}{}

\section{计算机技能}
\cvline{编程语言}{\small C(熟悉), C++, Java}
\cvline{脚本}{\small Python(熟悉), Sql, R, Bash Shell, \LaTeX{}}
\cvline{框架和库}{\small 科学库(了解numpy, scipy, matplotlib, pandas, scikit-learn), \small web框架(Flask, webapp2(Google App engine)), 图形库(了解d3.js)}
\cvline{大数据}{Hadoop(python streaming)}
\cvline{工具}{\small Emacs(常用), vim, Eclipse, git}
\cvline{操作系统}{\small Linux(Ubuntu Desktop)}
%\cvcomputer{Basic}{UML, HTML, C++, Tcl, System C} {}{}
%\cvcomputer{Intermediate}{C, Matlab, Assembly(Intel x86)}{}{}
%\cvcomputer{Expert}{VHDL, nML}{}{}

\section{其他兴趣}
\cvline{}{\small 我也热爱开源文化,听音乐, 游戏和运动.}


\closesection{}
\end{CJK}
\end{document}
